Due to needing to move on to more relevant work, this sub-project was ended prematurely. Hence, there is a notably long list of work that still needs to be done. I will separate these lists based on general category: \textbf{S}imulation \textbf{C}ode, \textbf{A}nalysis \textbf{S}cripts, and \textbf{S}ystem \textbf{E}quations. 

The simulation code feels fairly complete to me, although there are additions and improvements that could be made.

\begin{enumerate}[\textbf{SC}--1)]
	\item Possibly reformat the output files to be nicer to analyze. Not necessarily one run per file, but maybe a maximum of 100 runs per file.
	\item It is currently difficult to calculate the statistics for the runs that use adaptive time-stepping, as the step size may be adjusted differently for each run. It would be nice if each method ensure consistent time-steps between runs (e.g., use different time-step sizes to ensure information is output at the user-input time-step size).
	\item The scheme that uses the analytic statistics with the trapezoidal method for numeric integration seems very unstable. This may be due to the unboundedness of the variances and covariances of the system, but it may be an error in the code.
	\item The code has not yet been statistically analyzed, which is necessary if this code is to be used for other purposes.
\end{enumerate}

The evaluation scripts are currently written in Python, and automatically read in data from the output files \texttt{outXXX.nc} in the \texttt{build} subcdirectory in which the code is compiled and linked (as opposed the the \texttt{build} subdirectory for this documentation.

\begin{enumerate}[\textbf{AS}--1)]
	\item These scripts are incomplete as in the do not compare all of the statistics of the output of the simulations against the analytic formulas. The rest of the analytic formulas need to be implemented in an efficient way, e.g., calculate $\vr{\func{u}{t_{k+1}}}$ using $\vr{\func{u}{t_{k}}}$ instead of $\vr{\func{u}{t_{0}}}$.
	\item  It would be convenient if the scripts could read in the \texttt{NAMELIST} to obtain the simulation parameters, or at least have an option to do so (possibly with command-line arguments).
	\item A written apology at the top of each script for their general sloppiness, including (but not limited to) no standard order to input arguments into the many functions used to calculate the analytic statistics.
\end{enumerate}

Although we have analytic equations for the statistics of the system, there may be better ways to write them.

\begin{enumerate}[\textbf{SE}--1)]
	\item There are many equations in Appendix~\ref{app:exact_stats} which are specific cases of more general equations, e.g., $\vr{\func{J}{s,\ t}}$ (Eq.~\ref{eqn:var_Jst}) and $\cov{\func{J}{s,\ t}}{\func{J}{r,\ t}}$ (Eqs.~\ref{eqn:cov_Jst_Jrt_r_le_s} and \ref{eqn:cov_Jst_Jrt_s_le_r}). It would be nice to eliminate the redundant special cases, or at least their derivations. Although it is nice to have separate derivations of the simpler forms as a way to check the more general forms.
	\item It might be useful to define variables/functions for more common terms in the equations, to cut down on how long they can be.
	\item It would be useful to know which statistics depend on each other, for when someone wants to make a plot of the analytic statistics and needs the statistics at the previous time-step to get the statistics at the current time-step.
\end{enumerate}